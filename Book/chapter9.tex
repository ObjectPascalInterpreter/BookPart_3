{\bfseries\slshape\sffamily\color{ChapterTitleColor} \chapter{Reference to Libraries}} \label{chap:libraries}

\section{Introduction}

This chapter is a reference chapter for the current set of built-in modules.

\section{Lists Module and List Object Methods}

The list module is called {\tt lists}. If you type {\tt lists} at the console you'll get:

\begin{lstlisting}
>>lists
Module: lists List Module
\end{lstlisting}

We can use {\tt dir()} to see the current set of supported methods in the list module:

\begin{lstlisting}
>>lists.dir()
["rndu","range","dir","rndi"]
\end{lstlisting}

These methods are described below:

\colorbox{blue!10}{\bf range(start, end, step):} This method returns (end-start)/step values in a list between the values of start and end.

\begin{lstlisting}
>> lists.range (0, 10, 2)
[0,2,4,6,8]
\end{lstlisting}


\colorbox{blue!10}{\bf rndi(lower, upper, n):} This method returns a list random integer such that lower $\leq n \leq$ upper. The random numbers will include the upper and lower bounds.

\begin{lstlisting}
>>println (lists.rndi (5, 10, 3)
[10,6,7}
\end{lstlisting}


\colorbox{blue!10}{\bf rndu (n):} This method returns $n$ random floating point number in the range [0.0, 1.0).

\begin{lstlisting}
>>println (lists.rndu (4))
[0.651631591841578,0.56693290756084,0.532966091763228,0.318821681430563]
\end{lstlisting}


\subsection*{List Object Methods}

As well as methods from the list module, a list object itself has associated with it a number of methods. This list of methods can be obtained by using {\tt dir()} on a list object. For example:

\begin{lstlisting}
>> a = [1,2]
>> a.dir()
["len","find","toUpper","toLower","left","right","mid","trim","split","dir"]
\end{lstlisting}


\colorbox{blue!10}{\bf min ():} This returns the smallest value it can find in a list. The one restriction is the list must only contain numerical values. An error is reported if it is found to contain
a string, another list etc.


\begin{lstlisting}
>>a = [5,4,6,7,8]
>>a.min ()
5
\end{lstlisting}


\colorbox{blue!10}{\bf max ():} This returns the largest value it can find in a list. The one restriction is the list must only contain numerical values. An error is reported if it is found to contain
a string, another list etc.

\begin{lstlisting}
>>a = [5,4,6,7,8]
>>a.max ()
8
\end{lstlisting}


\colorbox{blue!10}{\bf append (value):} This method takes one argument, a value and will append the list. The method does not return anything but the associated list is modified in the process.
\begin{lstlisting}
>>a = [5,4,6,7,8]
>>a.append (23)
>>println (a)
[,4,6,7,8,23]
\end{lstlisting}

Note that if you want to preserve the original list, then make a copy first as in:

\begin{lstlisting}
>>a = [5,4,6,7,8]
>>b = a
>>a.append (23)
>>println (a)
[5,4,6,7,8,23]
>>println (b)
[5,4,6,7,8]
\end{lstlisting}


\colorbox{blue!10}{\bf pop ():} This removes the last item from the list and returns the item to the caller.

\begin{lstlisting}
>>a = [5,4,6,7,8]
>>a.remove ()
>>println (a)
[5,4,6,7]
\end{lstlisting}


\colorbox{blue!10}{\bf len ():} This returns length of the list.

\begin{lstlisting}
>>a = [5,4,6,[7,8], True]
>>println (a.len ())
5
\end{lstlisting}


\colorbox{blue!10}{\bf insert(value, insertIndex):} This is a method that will insert a new item before the indexth-position. Indexing starts at zero.
\begin{lstlisting}
>>a = [5,4,6,[7,8]]
>>a.insert (True, 2)
>>a
[5,4,6,True,[7,8]]
\end{lstlisting}


%\colorbox{blue!10}{\bf sum (list):} This returns the sum of the numeric values. If the list contains even a single item that is not a numerical value (integer or float) an error is issued.
%
%\begin{lstlisting}
%>>a = [5,4,6]
%>>println (a.sum ())
%15
%\end{lstlisting}


% --------------------------------------------------------------------------------------
\section{Strings Module and String Object Methods}


The string module is called {\tt strings}. If you type {\tt strings} at the console you'll get:


\begin{lstlisting}
>>strings
Module: strings String Module
\end{lstlisting}

Use {\tt dir()} we can see the current set of supported strings methods:

\begin{lstlisting}
>>strings.dir()
["format","str","dir","val"]
\end{lstlisting}


\colorbox{blue!10}{\bf format (str, format):} This method will format a string according to a format specification. This follows the format specification used by Delphi, see~\url{https://docwiki.embarcadero.com/Libraries/Sydney/en/System.SysUtils.Format}.

\begin{lstlisting}
>>>strings.format (1.23, "%4.1f")
 1.2
\end{lstlisting}


\colorbox{blue!10}{\bf val (value):} This method will convert an integer or float into a string,
\begin{lstlisting}
>>>strings.val (2.356)
2.356
\end{lstlisting}


\colorbox{blue!10}{\bf str (str):} This method will convert a string into an integer or float depending on the string.

\begin{lstlisting}
>>>strings.str ("0.23")
0.23
\end{lstlisting}


\subsection*{String Object Methods}


\colorbox{blue!10}{\bf left (n):} This method returns the $n$ left-most characters of the string.

\begin{lstlisting}
>>"12345".left (3)
123
\end{lstlisting}


\colorbox{blue!10}{\bf right (n):} This method returns the $n$ right-most characters of the string.

\begin{lstlisting}
>>"12345".right (3)
345
\end{lstlisting}


\colorbox{blue!10}{\bf mid (n, length):} This method returns a string of specified length, length, and starting point, $n$ from the given string. Note that $n$ indexes from zero.

\begin{lstlisting}
>>"123456789".mid (3, 5)
45678
\end{lstlisting}


\colorbox{blue!10}{\bf len ():} This method returns a string of specified length, length, and starting point, $n$ from the given string. Note that $n$ indexes from zero.

\begin{lstlisting}
>>"123456789".len ()
9
\end{lstlisting}


\colorbox{blue!10}{\bf trim ():} This method returns a string where any spaces on the left or right of the string have been removed.

\begin{lstlisting}
>>" 123456789 6  ".trim ()
123456789 6
\end{lstlisting}


\colorbox{blue!10}{\bf toUpper ():}  This method returns a string where any lowercase characters have been made uppercase.

\begin{lstlisting}
>>"aBcD".toUpper ()
ABCD
\end{lstlisting}
%

\colorbox{blue!10}{\bf toLower ():} This method returns a string where any uppercase characters have been made lowercase case.

\begin{lstlisting}
>>"aBcD".toLower ()
abcd
\end{lstlisting}


\colorbox{blue!10}{\bf var.find (substr):} This method finds a substring in string. It returns -1 if it fails or the indexth position if it succeeds. Note indexing starts at zero.

\begin{lstlisting}
>>"ABCDEFG".find ("CD")
2
\end{lstlisting}


\colorbox{blue!10}{\bf var.split (character):} This method will splits the string argument at a given character into a list of strings. Note the split character is not included in the split strings.

\begin{lstlisting}
>>>"AB CD DE".split (" ")
["AB","CD","DE"]
\end{lstlisting}


\section{Arrays Module and Array Object Methods}



\section{Math Library: math}

The math module is called {\tt math}. If you type {\tt math} at the console you'll get:

\begin{lstlisting}
>>math
Module: math Math Module
\end{lstlisting}

Using {\tt dir()} we can see the current set of supported strings methods:

\begin{lstlisting}
["min","e","cos","dir","toRadians","tan","round","exp","toDegrees","abs",
"ceil","atan","sin","log","max","pi","ln","asin","sqrt","acos","floor"]
\end{lstlisting}

Some of these should be self-explanatory such as {\tt sqrt} for computing the square root, and {\tt sin, cos} and {\tt tan} for computing the elementary trigonometric functions (arguments should be in radians). Akmost all of them appear with the same name in other programming languages

The {\tt acos, asin} and {\tt atan} are the inverse trigonometric functions.

{\tt toDegree} and {\tt toRadians} will convert radians and degrees to degrees and radians respectively.

\begin{lstlisting}
>>math.toDegrees (math.pi/2)
90
>>math.toRadians (180)
3.14159265358979
\end{lstlisting}

{\tt log} computes the logarithm to the based 10 and {\tt ln} the log to the base $e$. {\tt exp} will compute the exponential $e^x$.

The two constants {\tt pi} and {\tt e} return $\pi$ and Napier's constant, $e$ respectively.


{\tt min} will return the minimum of two numbers and {\tt max} the maximum of two numbers.

\begin{lstlisting}
>>math.min (4, 2)
2
>>math.max (7, 3.4)
7
\end{lstlisting}

{\tt abs} will return the absolute value.


\begin{lstlisting}
>>math.abs (-5.6)
5.6
\end{lstlisting}

{\tt round} will returns the value rounded to the nearest integer number.

\begin{lstlisting}
>>math.round (-5.6)
6
>>math.round (-5.1)
5
>>math.round (5.6)
6
>>math.round (5.1)
5
\end{lstlisting}


{\tt floor} will round down the nearest integer. That means $-3.4$ goes to $-3$ and $3.4$ goes to 2.

\begin{lstlisting}
>>math.floor (-5.1)
6
>>math.floor (5.9)
5
\end{lstlisting}

{\tt ceil}, short for ceiling, will round up to the nearest integer. That means $-3.9$ goes to $-3$ and $3.4$ goes to 3.

\begin{lstlisting}
>>math.ceil (-5.1)
5
>>math.ceil (5.9)
6
\end{lstlisting}


\section{Random Module: random}

The random module is called {\tt random} and provides access to random number generation. If you type {\tt random} at the console you'll get:

\begin{lstlisting}
>>random
Module: random Random Module
\end{lstlisting}

Use {\tt dir()} we can see the current set of supported strings methods:

\begin{lstlisting}
["randint","gauss","seed","dir","random"]
\end{lstlisting}

\colorbox{blue!10}{\bf random ():} Returns a pseudo-random floating point number between $0$ and $1$ (but not including 1). In this version the random number generator uses the Delphi random function which itself uses a simple linear congruential generator, that is a recursive equation of the form: $ X_{x+1} = (a X_n + c) \mod m$. According to Wikipedia (\url{https://en.wikipedia.org/wiki/Linear_congruential_generator}), Delphi {\tt random} uses a modulus, $m$ of $2^{23}$, a multiplier $a$ of $134775813$ and an increment $c$ of 1.

\begin{lstlisting}
>>random.random ()
0.49627447151579
\end{lstlisting}


\colorbox{blue!10}{\bf randint (number):} Returns an integer pseudo-random number between $0$ and $number - 1$

\begin{lstlisting}
>>random.randint (5)
2
\end{lstlisting}


\colorbox{blue!10}{\bf gauss (mean, stdev):} This method returns a pseudo-random number drawn from a normal distribution with mean $mean$ and standard deviation, $stdev$.

\begin{lstlisting}
>>random.gauss (1, 0.4)
0.532361312210798
\end{lstlisting}


\colorbox{blue!10}{\bf seed (number):} Use this method to set the seed for the random number generator. This allows stochastic studies to be erectly repeatable.  A convenient `random' seed can be provided by {\tt time.getTickCount()}.
\begin{lstlisting}
>>random.randint (5)
2
\end{lstlisting}


%\section{Useful Reading}
%
%\subsection{Introductory Books}
%

\begin{center}
\pgfornament[width = 8cm, color = cardinal]{83}
\end{center} 